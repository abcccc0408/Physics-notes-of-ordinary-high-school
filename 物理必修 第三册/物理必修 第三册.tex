\documentclass[12pt,a4paper]{ctexart}

\title{物理必修\ \ 第三册}
\author{啊波呲}

\setlength{\parskip}{0em}
\usepackage{amsmath,mathtools,amssymb,geometry,wrapfig,graphicx,empheq,pifont}
\renewcommand{\baselinestretch}{1.8}
\geometry{left=1.5cm,right=1.5cm,bottom=2.1cm,top=2.5cm}
\usepackage{tikz}
\usepackage{xcolor}
\newcounter{exam}[section]
\setcounter{exam}{0}
\newcommand{\bre}{\ \ \ }

\begin{document}
\maketitle
\pagenumbering{roman}
\tableofcontents

\newpage
\pagenumbering{arabic}

\setlength{\abovedisplayskip}{3pt}
\setlength{\belowdisplayskip}{3pt}

\section{静电场}
% \begin{empheq}[box=\fbox]{equation*}
%     1
% \end{empheq}

% \refstepcounter{exam}
% \subparagraph{例\theexam}

\subsection{电荷}

人们发现, 很多物体都会由于摩擦而带电, 并称这种方式为\textbf{摩擦起电}.

美国科学家富兰克林通过实验发
现, 雷电的性质与摩擦产生的电的性质完全相
同, 并命名了\textbf{正电荷}和\textbf{负电荷}.自然界的电荷只有两种.

电荷的多少叫作\textbf{电荷量}, 用$Q$或$q$表示. 在国际单位制中, 它的单位是\textbf{库
    仑}, 简称库, 符号是 C, 定义为1 A恒定电流在1 s 时间间隔内所传送的电荷量为1 C. 因此,
电荷量不属于基本物理量, 它是电流强度$I$和时间$t$的导出物理量, 并且$$Q = It$$.

\section{静电场}

19 世纪 30 年代, 英国科学家法拉第提出一种观点, 认
为在电荷的周围存在着由它产生的电场.处在电场中的其他电荷受到的作用力就是这个电场给
予的.例如, 电荷 A 对电荷 B 的作用力, 就是电荷 A 的电
场对电荷 B 的作用;电荷 B 对电荷 A 的作用力, 就是电荷 B
的电场对电荷 A 的作用.

物理学的理论和实验证实并发展了法拉第的观点.电场
以及磁场已被证明是客观存在的.场像分子、原子等实物
粒子一样具有能量, 因而场也是物质存在的一种形式.

静止电荷产生的电场叫作\textbf{静电场}.

把一个电荷放入某个电场中, 来研究这个电场的性质. 这样的
电荷叫作\textbf{试探电荷}。激发电场的带电体所带的电荷叫
作\textbf{场源电荷},或\textbf{源电荷}.

试探电荷应当是电荷量很小的点电荷, 目的是不对所研究的电场产生影响.

\subsection{电场强度}

试探电荷在电场中某点所受的电场力和它的电荷量的比值叫做该点的\textbf{电场强度}.

电场强度是矢量.物理学规定, 电场中某点的电场强度的方向与正电荷在该点所受静电力的方向相同.

在点电荷$Q$的电场中的$P$点,放一个试探电
荷$q_1$,它在电场中受到的静电力是$F_1$,根据库仑定律,有

\end{document}