\documentclass[12pt,a4paper]{ctexart}

\title{物理必修\ \ 第三册}
\author{啊波呲}

\setlength{\parskip}{0em}
\usepackage{amsmath,mathtools,amssymb,geometry,wrapfig,graphicx,empheq,pifont}
\renewcommand{\baselinestretch}{1.77}
\geometry{left=1.5cm,right=1.5cm,bottom=2.1cm,top=2.5cm}
\usepackage{tikz}
\usepackage{xcolor}
\newcounter{exam}[section]
\setcounter{exam}{0}
\newcommand{\bre}{\ \ \ }
\newcommand{\examlabel}{\textbf{例\theexam}}
\newcommand{\soln}{\textbf{解}\bre}
\newcommand{\notes}{\textbf{注意}\bre}
\newcommand{\unit}[1]{\ \mathrm{#1}}

\newenvironment{example}{\par\refstepcounter{exam}\examlabel\bre}{\par}
\newenvironment{solution}{\par\soln}{\par}

\begin{document}
\maketitle
\pagenumbering{roman}
\tableofcontents

\newpage
\pagenumbering{arabic}

\setlength{\abovedisplayskip}{3pt}
\setlength{\belowdisplayskip}{3pt}

\section{静电场}
% \begin{empheq}[box=\fbox]{equation*}
%     1
% \end{empheq}

% \refstepcounter{exam}
% \subparagraph{例\theexam}

\subsection{电荷}
\subsubsection{电荷}

人们发现, 很多物体都会由于摩擦而带电, 并称这种方式为\textbf{摩擦起电}.

美国科学家富兰克林通过实验发
现, 雷电的性质与摩擦产生的电的性质完全相
同, 并命名了\textbf{正电荷}和\textbf{负电荷}.自然界的电荷只有两种.

电荷的多少叫做\textbf{电荷量}, 用$Q$或$q$表示. 在国际单位制中, 它的单位是\textbf{库
    仑}, 简称库, 符号是 C, 定义为1 A恒定电流在1 s 时间间隔内所传送的电荷量为1 C. 因此,
电荷量不属于基本物理量, 它是电流强度$I$和时间$t$的导出物理量, 并且$$Q = It.$$

正电荷的电荷量为正值, 负电荷的电荷量为负值.

\subsubsection{起电}

起电, 就是使物体带电. 起电的本质是电子转移.

摩擦可以使物体带电, 那么, 还有其它方法可以使物体带电吗?

\subparagraph{感应起电} 当一个带电体靠近导体时, 由于电荷间相互吸引或排
斥, 导体中的自由电荷便会趋向或远离带电体, 使导体靠
近带电体的一端带异种电荷, 远离带电体的一端带同种电
荷.这种现象叫做\textbf{静电感应}.利用
静电感应使金属导体带电的过程叫做\textbf{感应起电}.

与摩擦起电不同的是, 摩擦起电的两个物体通常都是绝缘体, 这使电荷会留在绝缘体表面, 产生明显的带电现象.相反,导体的电荷会均匀分布整个物体,
不易观察到带电现象.

\subparagraph{接触起电} 感应起电的两个物体不接触, 电荷仅在导体的内部移动.
除此之外, 两个物体在接触时, 如果它们之间存在电位差, 电荷将会在两个物体间转移,
最终达到动态平衡.

一个不带电的导体通过与一个带电体接触后分开, 从而形成带电体的过程, 称为\textbf{接触起电}.

两个完全相同的导体接触后分开, 它们所带的电荷量相同.因此, 除非这两个物体都不带电, 否则在
接触后将会相互排斥.

\subsubsection{电荷守恒定律}

静电感应过程中导体中的自由电荷只是从导体的一部
分转移到另一部分. 而接触起电过程中自由电荷在几个导体间转移.
也就是说, 无论是接触起电还是感应起电都没有创造电荷, 只是电荷的分布发生了变化.

大量实验事实表明, \textbf{电荷既不会创生, 也不会消灭, 它
    只能从一个物体转移到另一个物体, 或者从物体的一部分转
    移到另一部分;在转移过程中, 电荷的总量保持不变.}这个
结论叫做\textbf{电荷守恒定律}.

电荷守恒定律更普遍的表述是:\textbf{一个与外界没有电荷交换的系统, 电
    荷的代数和保持不变.}

\subsubsection{元电荷}
迄今为止, 实验发现的最小电荷量就是电子所带的电
荷量.质子, 正电子所带的电荷量与它相同, 电性相反.
人们把这个最小的电荷量叫做\textbf{元电荷}, 用$e$表示.

元电荷$e$的数值, 最早是由美国物理学家密立根测得
的.在密立根实验之后, 人们又做了许多测量.现在公认的元电荷$e$的值为
$$e = 1.602 176 634 \times 10^{-19}\ \mathrm{C}.$$
在计算中, 可取
$$e = 1.60 \times 10^{-19}\ \mathrm{C}.$$

电子的电荷量$e$与电子的质量$m_e$之比, 叫做电子的\textbf{比
    荷}.比荷也是一个重要的物理量.电子的
质量$m_e = 9.11\times 10^{-31}$ kg, 所以电子的比荷为
$$\frac{e}{m_e} = 1.76 \times 10^{11}\ \mathrm{C/kg}.$$

\subsection{静电力}

通过实验可知, 电荷之间的作用力随着电
荷量的增大而增大, 随着距离的增大而减小.
这看起来与万有引力的规律类似.电荷之间的相互作用力, 会不会与它们电
荷量的乘积成正比, 与它们之间距离的二次方成反比?

法国科学家库仑设计了一个十分精妙的实验(扭秤实验), 对电荷之间的作
用力开展研究, 最终确定: \textbf{真空中两个静止点电荷之间的
    相互作用力, 与它们的电荷量的乘积成正比, 与它们的距
    离的二次方成反比, 作用力的方向在它们的连线上.}这个
规律叫做\textbf{库仑定律}.这种电荷之间的相
互作用力叫做\textbf{静电力}或\textbf{库仑力}.

假设两个点电荷的电荷量的绝对值分别为$q_1$, $q_2$, 它们的距离为$r$, 那么库仑定律可以表示为
\begin{empheq}[box=\fbox]{equation*}
    F = k\frac{q_1q_2}{r^2}.
\end{empheq}
式中的$k$是比例系数, 叫作\textbf{静电力常量}.当两个点电荷所带
的电荷量为同种时, 它们之间的作用力为斥力; 反之, 为
异种时, 它们之间的作用力为引力.
在国际单位制中, 电荷量的单位是库仑(C), 力的单
位是牛顿(N), 距离的单位是米(m). $k$的值是
$$k = 9.0 \times 10^9\ \mathrm{N\cdot m^2/C^2}.$$

上面的定律中提到了点电荷的概念, 下面我们来介绍一下.

\subparagraph{点电荷} 实验事实说明, 两个实际的带电体间的相互作用力与
它们自身的大小、形状以及电荷分布都有关系.

当带电体之间的距离比它们自身的大小大得多, 以致带电体的形状, 大小及电荷分布状
况对它们之间的作用力的影响可以忽略时, 这样的带电体可以看作带电的点, 叫做\textbf{点电荷}.
\\

库仑定律描述的是两个点电荷之间的作用力.如果存
在两个以上点电荷, 那么, 每个点电荷都要受到其他所有
点电荷对它的作用力. 两个或两个以上点电荷对某一个点
电荷的作用力, 等于各点电荷单独对这个点电荷的作用力
的矢量和.

库仑定律是电磁学的基本定律之一.库仑定律给出的
虽然是点电荷之间的静电力, 但是任何一个带电体都可以
看成是由许多点电荷组成的. 所以, 如果知道带电体上的
电荷分布, 根据库仑定律就可以求出带电体之间的静电力
的大小和方向.

\subsection{电场}

\subsubsection{电场}

19 世纪 30 年代, 英国科学家法拉第提出一种观点, 认
为在电荷的周围存在着由它产生的电场.处在电场中的其它电荷受到的作用力就是这个电场给
予的.例如, 电荷 A 对电荷 B 的作用力, 就是电荷 A 的电
场对电荷 B 的作用;电荷 B 对电荷 A 的作用力, 就是电荷 B
的电场对电荷 A 的作用.

物理学的理论和实验证实并发展了法拉第的观点.电场
以及磁场已被证明是客观存在的.场像分子, 原子等实物
粒子一样具有能量, 因而场也是物质存在的一种形式.

静止电荷产生的电场叫做\textbf{静电场}.

把一个电荷放入某个电场中, 来研究这个电场的性质. 这样的
电荷叫做\textbf{试探电荷}.激发电场的带电体所带的电荷叫
作\textbf{场源电荷}, 或\textbf{源电荷}.

在研究电场的性质时, 我们选取的试探电荷应当是电荷量很小的点电荷, 目的是不对所研究的电场产生影响.

\subsubsection{电场强度}

在点电荷$Q$的电场中的$P$点, 放一个试探电荷$q_1$, 它在电场中受到的静电力是$F_1$, 根据库仑定律, 有
\begin{equation}
    F_1 = k\frac{q_1Q}{r^2}.
    \label{静电力1}
\end{equation}

同理, 如果把试探电荷换成$q_2$, 那么它受到的静电力
\begin{equation}
    F_2 = k\frac{q_2Q}{r^2}.
    \label{静电力2}
\end{equation}

由 \eqref{静电力1} \eqref{静电力2} 两式可以看出
\begin{equation}
    \label{静电力与电荷量之比}
    \frac{F_1}{q_1} = \frac{F_2}{q_2} = k\frac{Q}{r^2}.
\end{equation}
放在$P$点的试探电荷所受的静电力与它的电荷量之比, 与产生电场的场源电荷
的电荷量 $Q$及$P$点到场源电荷的距离$r$有关, 而与试探电荷的电荷量无关.

试探电荷所受的静电力与它的电荷量之比反映了电场在各点的性质. 我们定义它为\textbf{电场强度},
用$E$表示, 即
\begin{empheq}[box=\fbox]{equation*}
    E = \frac{F}{q}.
\end{empheq}
这是电场强度的定义式.其中$F$是试探电荷在电场内某点所受的静电力, $q$是这个试探电荷的电荷量, $E$是这一点的电场强度.

由定义式可知, 电场强度的国际单位为\textbf{牛每库}, 符号是N/C.如果 1 C 的电荷在电场中的某点受到的静电力是 1 N,那么
该点的电场强度就是 1 N/C.

电场强度是矢量.物理学规定, 电场中某点的电场强度的方向与正电荷在该点所受静电力的方向相同.

试探电荷在电场中受到的静电力也叫做\textbf{电场力}.

\subsubsection*{点电荷的电场强度}

点电荷是最简单的场源电荷.由 \eqref{静电力与电荷量之比} 可知,
一个电荷量为$Q$的点电
荷, 在与之相距$r$处的电场强度
\begin{empheq}[box=\fbox]{equation}
    E = k\frac{Q}{r^2}.
    \label{点电荷的电场强度}
\end{empheq}

据上式可知, 如果以电荷量为$Q$的点电荷为中心作一
个球面, 则球面上各点的电场强度大小相等. 当$Q$为正电荷
时, 电场强度$E$的方向沿半径向外; 当$Q$为负
电荷时, 电场强度$E$的方向沿半径向内.

\subsubsection*{电场强度的叠加} 我们知道, 两个或两个以上的点电荷对某一个点电荷
的静电力, 等于各点电荷单独对这个点电荷的静电力的矢
量和.由此可以推理, 如果场源是多个点电荷, 则电场中
某点的电场强度等于各个点电荷单独在该点产生的电场强
度的矢量和.
\\

\begin{example}
    在某电场中的$P$点, 放一带电量$q_1 = -3.0\times 10^{-10}\unit{C}$的试探电荷,
    测得该点收到的静电力大小为$F_1 = 6.0\times 10^{-7}\unit{N}$, 方向水平向右. 求

    (1) $P$点的电场强度大小和方向;

    (2) 如果在$P$点放一带电量$q_2 = 1.0 \times 10^{-10}\unit{C}$的试探电荷, 求$q_2$
    受到的静电力$F_2$的大小和方向.
\end{example}
\begin{solution}
    (1) 根据电场强度的定义, $P$点的电场强度为
    $$E = \frac{F_1}{q_1} = \frac{6.0\times 10^{-7}}{-3.0\times 10^{-10}}\unit{N/C} = 2.0\times 10^{3}\unit{N/C},$$
    方向与负点电荷$q_1$受到的静电力方向相反, 即水平向左.

    (2)由电场强度的定义有$$E = \displaystyle\frac{F_2}{q_2}.$$
    由此可得$$F_2 = 2.0\times 10^{-7}\unit{N}.$$因为$q_2$是正点电荷, 所以
    $F_2$的方向与$P$点的场强方向相同,即水平向左.
\end{solution}

\subsubsection{电场线}

除了用数学公式描述电场外, 形象地了解和描述电场中
各点电场强度的大小和方向也很重要.
法拉第采用了一个简洁的方法来描述电场, 那就是画\textbf{电场线}.

同一幅图中, 电场强度较大的地方电场线较密, 电场强度
较小的地方电场线较疏, 因此在同一幅图中可以用电场线
的疏密来比较各点电场强度的大小.

电场线有以下两个特点:

(1) 电场线从正电荷或无限远出发, 终止于无限远或
负电荷;

(2) 电场线在电场中不相交, 这是因为在电场中任意
一点的电场强度不可能有两个方向.

\subsubsection*{匀强电场}

如果电场中各点的电场强度的大小相等,方向相同,
这个电场就叫作\textbf{匀强电场}.

由于方向相同, 匀强电场中的
电场线应该是平行的; 又由于电场强度大小相等, 电场线
的疏密程度应该是相同的.所以, 匀强电场的电场线可以
用间隔相等的平行线来表示.

\subsection{静电场中的能量}

我们知道, 电荷在电场中会受到静电力, 若电荷发生位移, 则静电力可能会做功. 我们可以以此为突破,
了解电场中的能量.

\subsubsection*{静电力做功的特点}

实验发现, 将一个试探电荷$q$从$A$点移动到$B$点, 无论是沿直线移动, 还是沿折线或曲线移动, 静电力做
的功都相等.这就是说, 在静电场中移动电荷时, 静电力所做的功只与电荷的初末位置有关, 而与电荷经过的路径
无关. 因此, 与重力一样, \textbf{静电力属于保守力}.

我们知道, 静电力对电荷做的功应该等于静电力的大小, 电荷的位移大小, 静电力与位移夹角的余弦值这三者的乘积.
即$$W = Fl\cos \theta = qEl\cos \theta.$$

\subsubsection{电势能}
与物体在重力场中具有重力势能类似, 电荷在静电场中具有\textbf{电势能}, 用$E_p$表示.

如果用$W_{AB}$表示电荷由 $A$ 点运动到 $B$ 点静电力所做的
功, 用$E_{\mathrm{p}A}$表示电荷在$A$点所具有的电势能, 用$E_{\mathrm{p}B}$表示电荷在$B$点所具有的电势能,
那么它们之间的关系为
$$W_{AB} = E_{\mathrm{p}A} - E_{\mathrm{p}B}.$$
也可以表示为
$$W_{AB} = -\Delta E_\mathrm{p}.$$

当$W_{AB}>0$时, $E_{\mathrm{p}A} > E_{\mathrm{p}B}$, 静电力做正功, 电势能减小;

当$W_{AB}<0$时, $E_{\mathrm{p}A} < E_{\mathrm{p}B}$, 静电力做负功, 电势能增大.

这满足保守力做功的规律.

\subparagraph{电势能的相对性} 静电力做的功只能决定电势能的变化量,
而不能决定电荷在电场中某点电势能的数值.只有先把电
场中某点的电势能规定为 0, 才能确定电荷在电场中其他
点的电势能.

通常,我们把电荷在离场源电荷无限远处的电势能规定为 0,
或把电荷在大地表面的电势能规定为 0.
\\

规定离场源电荷$Q$无限远处的电势能为 0, 若将一个电荷$q$从$A$点移动到无限远处,
根据$W_{AB} = E_{\mathrm{p}A} - E_{\mathrm{p}B}$可得$$W_{A\rightarrow \infty} = E_{\mathrm{p}A} - 0 =E_{\mathrm{p}A}.$$
由此可知, \textbf{如果规定无限远处的电势能为 0, 那么
    电场中某电荷的电势能, 等于将该电荷从该点移动到无穷远处静电力所做的功}.

如果场源电荷$Q$是点电荷, $A$点到场源电荷的距离为$r$, 根据库仑定律,
通过数学推导可以得到
$$E_{\mathrm{p}A} = W_{A\rightarrow \infty} = Fl = k\frac{Qq}{r}.$$
式中$l$表示$A$点到无限远处的距离, $k$是静电力常量. 这里$Q$与$q$的正负要代入计算.
推导过程涉及高等数学知识, 故不做展开.

由此可得, 如果规定无限远处的电势能为 0, 那么在点电荷$Q$的电场中的某点, 电荷$q$所具有的电势能为
\setlength{\abovedisplayskip}{10pt}
\setlength{\belowdisplayskip}{10pt}
\begin{empheq}[box=\fbox]{equation}
    E_\mathrm{p} = k\frac{Qq}{r}.
    \label{点电荷附近的电势能}
\end{empheq}
式中, $k$是静电力常量, $r$是场源点电荷$Q$与试探电荷$q$的距离.
\\

应该注意, 电势能是相互作用的电荷所共有
的, 或者说是电荷及对它作用的电场所共有的.我们刚才说某
个电荷的电势能, 只是一种简略的说法.

\subsubsection{电势}

前面我们通过对静电力的研究, 认识了电场强度. 试探电荷在电场中所受的静电力与
试探电荷和场源电荷均有关; 而电场强度与试探电荷无关, 只与场源电荷有关, 是电场本身的性质.

现在我们要通过对电势能的研究来认识另一个物理量——电
势,它同样是表征电场性质的重要物理量.

通过实验可知, 置于某一点的试探电荷$q$, 如果它的电荷
量变为原来的$n$倍,其电势能也变为原来的$n$倍. 电势能
与电荷量之比却是一定的, 它是由电场中该点的性质决定的, 与试探电荷本身无关.

与电场强度的定义类似, 试探电荷在电场中某一点的电势能与它的电荷量之比, 叫做
电场在这一点的\textbf{电势}. 如果用$\varphi$表示电势, 用$E_\mathrm{p}$表示试探电荷$q$的电势能,
则
\begin{empheq}[box=\fbox]{equation*}
    \varphi = \frac{E_\mathrm{p}}{q}.
\end{empheq}

在国际单位制中, 电势的单位是\textbf{伏特}, 符号是V.
在电场中的某一点, 如果电荷量为 1 C 的电荷在这点的电势
能是 1 J, 这一点的电势就是 1 V, 即
1 V = 1 J/C.

假如正的试探电荷沿着电场线的方向向外移动(场源电荷是正电荷),
它的电势能是逐渐减少的. 可以说, \textbf{沿着电
    场线方向电势逐渐降低}.

假如负的试探电荷沿着电场线的方向向外移动(场源电荷是正电荷),
那么它的电势能的绝对值是逐渐减少的; 而它的电势能是负值, 所以电势能是逐渐增加的.

与电势能的情况相似, 应该先规定电场中某处的电势
为 0, 然后才能确定电场中其他各点的电势.在规定了零电势点之后, 电场中各点的电势可以是正
值, 也可以是负值.

当规定无穷远处为零电势点时, 如果场源电荷是点电荷, 根据点电荷附近附近某点的电势能公式 \eqref{点电荷附近的电势能} 可知
\begin{empheq}[box=\fbox]{equation*}
    \varphi = k\frac{Q}{r}.
\end{empheq}
这是\textbf{点电荷的电场内某点的电势计算公式}. 其中$k$为静电力常量, $Q$为场源电荷的电荷量,
$r$为这一点到场源电荷的距离.

由上式可知, \textbf{如果规定无穷远处为零电势点, 那么正电荷附近的电势大于0, 负电荷附近的电势小于0}.

电势只有大小, 没有方向, 是个标量.

\subsubsection*{电势叠加原理}
我们知道, 场源电荷是多个点电荷时, 电场中
某点的电场强度等于各个点电荷单独在该点产生的电场强
度的矢量和.

与电场强度类似,
\textbf{多个点电荷在空间某点产生电场的电势, 为每个点电荷在该点产生电势的代数和.}
这就是电势叠加原理.

\subsubsection{电势差}

\subsubsection{等势面}

\subsection{电势差与电场强度的关系}

\end{document}